\documentclass[a4j]{jarticle}
\usepackage{color}
\usepackage{mflogo}
\font\marksheet=msheet
\font\mrksht=msheet2

\definecolor{gray9}{gray} {0.8} 

\newsavebox{\ichikarazero}
\savebox{\ichikarazero}{
{\marksheet 1}\hskip2pt{\marksheet 2}\hskip2pt%
{\marksheet 3}\hskip2pt{\marksheet 4}\hskip2pt%
{\marksheet 5}\hskip2pt{\marksheet 6}\hskip2pt%
{\marksheet 7}\hskip2pt{\marksheet 8}\hskip2pt%
{\marksheet 9}\hskip2pt{\marksheet 0}}

\newsavebox{\minuskarakyu}
\savebox{\minuskarakyu}{
{\marksheet -}\hskip5pt{\marksheet ?}\hskip5pt%
{\marksheet 0}\hskip5pt%
{\marksheet 1}\hskip5pt{\marksheet 2}\hskip5pt%
{\marksheet 3}\hskip5pt{\marksheet 4}\hskip5pt%
{\marksheet 5}\hskip5pt{\marksheet 6}\hskip5pt%
{\marksheet 7}\hskip5pt{\marksheet 8}\hskip5pt%
{\marksheet 9}\hskip5pt{\marksheet |}}

\newsavebox{\minuskarag}
\savebox{\minuskarag}{
{\marksheet -}\hskip5pt%
{\marksheet 0}\hskip5pt%
{\marksheet 1}\hskip5pt{\marksheet 2}\hskip5pt%
{\marksheet 3}\hskip5pt{\marksheet 4}\hskip5pt%
{\marksheet 5}\hskip5pt{\marksheet 6}\hskip5pt%
{\marksheet 7}\hskip5pt{\marksheet 8}\hskip5pt%
{\marksheet 9}\hskip5pt%
{\marksheet A}\hskip5pt{\marksheet B}\hskip5pt%
{\marksheet C}\hskip5pt{\marksheet D}\hskip5pt%
{\marksheet E}\hskip5pt{\marksheet F}\hskip5pt%
{\marksheet G}}

\newsavebox{\minuskarad}
\savebox{\minuskarad}{
{\marksheet -}\hskip5pt{\marksheet 0}\hskip5pt%
{\marksheet 1}\hskip5pt{\marksheet 2}\hskip5pt%
{\marksheet 3}\hskip5pt{\marksheet 4}\hskip5pt%
{\marksheet 5}\hskip5pt{\marksheet 6}\hskip5pt%
{\marksheet 7}\hskip5pt{\marksheet 8}\hskip5pt%
{\marksheet 9}\hskip5pt%
{\marksheet a}\hskip5pt{\marksheet b}\hskip5pt%
{\marksheet c}\hskip5pt{\marksheet d}}

\newsavebox{\ichizero}
\savebox{\ichizero}{
{\mrksht 1}\hskip5.4pt{\mrksht 2}\hskip5.4pt%
{\mrksht 3}\hskip5.4pt{\mrksht 4}\hskip5.4pt%
{\mrksht 5}\hskip5.4pt{\mrksht 6}\hskip5.4pt%
{\mrksht 7}\hskip5.4pt{\mrksht 8}\hskip5.4pt%
{\mrksht 9}\hskip5.4pt{\mrksht 0}}


\begin{document}

\unitlength=1mm
\medskip

\begin{center}
\begin{picture}(50,100)
\thicklines

\multiput(-20,15)(0,0){1}{\line(0,1){68}}  %太い縦線
\multiput(0,15)(0,0){1}{\line(0,1){68}}  %太い縦線
\multiput(20,15)(0,0){1}{\line(0,1){68}}  %太い縦線
\multiput(110,15)(0,0){1}{\line(0,1){68}}  %太い縦線

\multiput(-20,15)(0,68){2}{\line(1,0){130}} %太い横線 上下2本ok!

\multiput(-20,73)(0,15){1}{\line(1,0){130}} %太い横線クラス番号、学籍番号の下
\multiput(-20,68)(0,15){1}{\line(1,0){130}} %太い横線 十位 ... 英字の下

\thinlines
\multiput(-10,15)(10,0){13}{\line(0,1){58}} %縦棒

\multiput(-15,59)(10,0){1}{\makebox(0,0){理}}
\multiput(-15,54)(10,0){1}{\makebox(0,0){文}}


\multiput(-5,59)(10,0){1}{\makebox(0,0){\textcolor{gray9}{\marksheet 1}}}
\multiput(-5,54)(10,0){1}{\makebox(0,0){\textcolor{gray9}{\marksheet 2}}}
\multiput(-5,49)(10,0){1}{\makebox(0,0){\textcolor{gray9}{\marksheet 3}}}

\multiput(5,64)(10,0){8}{\makebox(0,0){\textcolor{gray9}{\marksheet 0}}}
\multiput(5,59)(10,0){8}{\makebox(0,0){\textcolor{gray9}{\marksheet 1}}}
\multiput(5,54)(10,0){8}{\makebox(0,0){\textcolor{gray9}{\marksheet 2}}}
\multiput(5,49)(10,0){8}{\makebox(0,0){\textcolor{gray9}{\marksheet 3}}}
\multiput(5,44)(10,0){8}{\makebox(0,0){\textcolor{gray9}{\marksheet 4}}}
\multiput(5,39)(10,0){8}{\makebox(0,0){\textcolor{gray9}{\marksheet 5}}}
\multiput(5,34)(10,0){8}{\makebox(0,0){\textcolor{gray9}{\marksheet 6}}}
\multiput(5,29)(10,0){8}{\makebox(0,0){\textcolor{gray9}{\marksheet 7}}}
\multiput(5,24)(10,0){8}{\makebox(0,0){\textcolor{gray9}{\marksheet 8}}}
\multiput(5,19)(10,0){8}{\makebox(0,0){\textcolor{gray9}{\marksheet 9}}}

\put(85,64){\makebox(0,0){\textcolor{gray9}{\marksheet A}}}
\put(85,59){\makebox(0,0){\textcolor{gray9}{\marksheet B}}}
\put(85,54){\makebox(0,0){\textcolor{gray9}{\marksheet C}}}
\put(85,49){\makebox(0,0){\textcolor{gray9}{\marksheet D}}}
\put(85,44){\makebox(0,0){\textcolor{gray9}{\marksheet E}}}
\put(85,39){\makebox(0,0){\textcolor{gray9}{\marksheet F}}}
\put(85,34){\makebox(0,0){\textcolor{gray9}{\marksheet G}}}
\put(85,29){\makebox(0,0){\textcolor{gray9}{\marksheet H}}}
\put(85,24){\makebox(0,0){\textcolor{gray9}{\marksheet I}}}
\put(85,19){\makebox(0,0){\textcolor{gray9}{\marksheet J}}}

\put(95,64){\makebox(0,0){\textcolor{gray9}{\marksheet K}}}
\put(95,59){\makebox(0,0){\textcolor{gray9}{\marksheet L}}}
\put(95,54){\makebox(0,0){\textcolor{gray9}{\marksheet M}}}
\put(95,49){\makebox(0,0){\textcolor{gray9}{\marksheet N}}}
\put(95,44){\makebox(0,0){\textcolor{gray9}{\marksheet O}}}
\put(95,39){\makebox(0,0){\textcolor{gray9}{\marksheet P}}}
\put(95,34){\makebox(0,0){\textcolor{gray9}{\marksheet Q}}}
\put(95,29){\makebox(0,0){\textcolor{gray9}{\marksheet R}}}
\put(95,24){\makebox(0,0){\textcolor{gray9}{\marksheet S}}}
\put(95,19){\makebox(0,0){\textcolor{gray9}{\marksheet T}}}

\put(105,64){\makebox(0,0){\textcolor{gray9}{\marksheet U}}}
\put(105,59){\makebox(0,0){\textcolor{gray9}{\marksheet V}}}
\put(105,54){\makebox(0,0){\textcolor{gray9}{\marksheet W}}}
\put(105,49){\makebox(0,0){\textcolor{gray9}{\marksheet X}}}
\put(105,44){\makebox(0,0){\textcolor{gray9}{\marksheet Y}}}
\put(105,39){\makebox(0,0){\textcolor{gray9}{\marksheet Z}}}


\put(-15,70.5){\makebox(0,0){\bf 科}}
\put(-5,70.5){\makebox(0,0){\bf 類}}
\put(5,70.5){\makebox(0,0){\bf 十位}}
\put(15,70.5){\makebox(0,0){\bf 一位}}
\put(25,70.5){\makebox(0,0){\bf 十万位}}
\put(35,70.5){\makebox(0,0){\bf 万位}}
\put(45,70.5){\makebox(0,0){\bf 千位}}
\put(55,70.5){\makebox(0,0){\bf 百位}}
\put(65,70.5){\makebox(0,0){\bf 十位}}
\put(75,70.5){\makebox(0,0){\bf 一位}}
\put(85,70.5){\makebox(0,0){\bf 英字}}
\put(95,70.5){\makebox(0,0){\bf 英字}}
\put(105,70.5){\makebox(0,0){\bf 英字}}

\put(-10,78){\makebox(0,0){\bf 科 類}}
\put(10,78){\makebox(0,0){\bf クラス番号}}
\put(65,78){\makebox(0,0){\bf 学 籍 番 号}}

\put(31,12){\makebox(0,0){\bf 注意:科に丸、学籍番号、クラス番号を鉛筆などで塗りつぶすこと。}}
\put(47,8){\makebox(0,0){\bf 記載がない場合、読み取りにくい場合は出席とならない場合がある。毎回確認すること。}}

\end{picture}
\end{center}

\end{document}
